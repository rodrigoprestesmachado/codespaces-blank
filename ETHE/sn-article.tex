\documentclass[sn-apa]{sn-jnl} % APA Reference Style

%%%% Standard Packages

\usepackage{graphicx}%
\usepackage{multirow}%
\usepackage{amsmath,amssymb,amsfonts}%
\usepackage{amsthm}%
\usepackage{mathrsfs}%
\usepackage[title]{appendix}%
\usepackage{xcolor}%
\usepackage{textcomp}%
\usepackage{manyfoot}%
\usepackage{booktabs}%
\usepackage{algorithm}%
\usepackage{algorithmicx}%
\usepackage{algpseudocode}%
\usepackage{listings}%


\begin{document}

\title[Article Title]{Analyzing Students Use Metacognitive Strategies When
Interacting with GenAI-Powered Chatbots in Programming Courses.}

%%=============================================================%%
%% GivenName	-> \fnm{Joergen W.}
%% Particle	-> \spfx{van der} -> surname prefix
%% FamilyName	-> \sur{Ploeg}
%% Suffix	-> \sfx{IV}
%% \author*[1,2]{\fnm{Joergen W.} \spfx{van der} \sur{Ploeg} 
%%  \sfx{IV}}\email{iauthor@gmail.com}
%%=============================================================%%

\author*[1]{\fnm{Rodrigo} \sur{Prestes Machado}}\email{rodrigo.prestes@poa.ifrs.edu.br}
\author*[2]{\fnm{Carlos} \sur{Alario-Hoyos}}\email{calario@it.uc3m.es}
\author[2]{\fnm{Iria} \sur{Estévez-Ayres}}\email{ayres@it.uc3m.es}
\author[2]{\fnm{Patricia} \sur{Callejo}}\email{pcallejo@it.uc3m.es}
\author[2]{\fnm{Carlos} \sur{Delgado Kloos}}\email{cdk@it.uc3m.es}

\equalcont{These authors contributed equally to this work.}

\affil*[1]{\orgdiv{Informatics}, \orgname{Federal Institute of Education,
Science and Technology}, \orgaddress{\street{Cel. Vicente}, \city{Porto Alegre},
\postcode{90.030-041}, \state{Rio Grande do Sul}, \country{Brazil}}}

\affil[2]{\orgdiv{Department of Telematics Engineering},
\orgname{Universidad Carlos III}, \orgaddress{ \street{Av. de la Universidad},
\city{Leganés}, \postcode{28911}, \state{Madrid}, \country{Spain}}}

%%==================================%%
%% Sample for unstructured abstract %%
%%==================================%%

\abstract{ This study aimed to analyze prompts of programming students with a
chatbot powered by OpenAI’s GPT-3.5, enhanced with the Retrieval Augmented
Generation (RAG) technique, within the context of a Java programming course. The
focus was on students using two metacognitive strategies: interleaving and
spacing. Student prompts were categorized into eight categories along with their
respective study topics. Findings revealed that the categories and markers of
spacing and interleaving were important in identifying study sessions with the
chatbot. However, students showed limited intentional application of these
learning strategies. These results highlight the need for more comprehensive
guidance on leveraging AI tools to improve learning outcomes.
}

\keywords{Programming, Generative Artificial Intelligence, Chatbots, Spacing,
Interleaving and Metacognition}

\maketitle

\section{Introduction}\label{sec1}

Recent advances in Generative Artificial Intelligence (GenAI) have created new
opportunities in professional programming and education \citep{Puryear22}. In
programming courses, students can employ GenAI tools to improve their
understanding, receive personalized feedback, and access detailed explanations.
For example, GitHub Copilot, a tool integrated into development environments,
assists in providing real-time suggestions and accelerating code writing, which
could help streamline the learning process. On the other hand, educational
chatbots try to guide students through coding challenges, answer questions, and
foster a deeper understanding of programming concepts.

The study by \cite{chan23} revealed that undergraduate and postgraduate students
have positive attitudes towards using GenAI. A systematic review organized by
\cite{Lo24} demonstrated that students could effectively learn from ChatGPT,
resulting in improved comprehension and academic achievement \citep{Callejo24}.
Additionally, it was observed that ChatGPT can enable students to control their
learning pace and may support self-regulated learning, especially for those with
prior technical and disciplinary knowledge \citep{Xia23}.

However, researchers have also expressed concerns about the impact of these
tools on students. The systematic review by \cite{Murillo23} indicated that the
use of ChatGPT could lead to overreliance on the tool. \cite{chan23} noted that
overdependence on ChatGPT could lead to a decrease in critical thinking, as
students can make decisions based solely on the information provided by the
tool. As a potential consequence, \cite{Bastani24} observed in a study that when
programming students lost access to ChatGPT, those who had previously relied
on it saw their performance drop by 17\%. In contrast, students who had never
used the tool were unaffected and outperformed their peers.

The confidence of students in these tools is well founded, as shown by
\cite{Puryear22}, who found that GitHub Copilot can generate solutions for
student assignments with precision rates ranging from 68\% to 95\%. However,
\cite{Boudouaia24} observed that using ChatGPT without a structured learning
approach did not provide significant improvement over traditional self-directed
learning methods in programming. This raises concerns that students may become
overly dependent on generative AI tools, potentially hindering their learning
progress and missing opportunities to develop a deeper understanding of
fundamental concepts.

Regardless of teachers' preferences or beliefs, preliminary surveys conducted
by \cite{Dickey24} indicate that more than 54.5\% of students are already using
GenAI for homework, likely due to their perception of this technology as
advantageous and intuitive \citep{Boudouaia24}. All of these studies highlight
the need to increase the understanding of how students interact with these tools
and how they can be used to improve learning, as emphasized by \cite{Lo24}.

In response to the growing need for deeper insight into the use of GenAI tools
in educational settings, this study aims to help address this gap by analyzing
the interactions between students in a Java programming course and an
educational chatbot powered by OpenAI gpt-3.5, enhanced with the
Retrieval-Augmented Generation (RAG) technique. Specifically, it focuses on
examining these interactions within the context of two metacognitive strategies:
spacing \citep{Carvalho20} and interleaving \citep{Rivers21}. To achieve this
objective, we formulate three research questions:

\begin{itemize}
    \item RQ1 - What distribution patterns emerged in the classified student
    interaction prompts with the educational chatbot?
    \item RQ2 - Was there spacing between student prompts? Which category led to
    the most rapid and consistent interactions with the chatbot?
    \item RQ3 - Do students' interactions with the chatbot alternate between study
    topics to suggest interleaving?
  \end{itemize}

This paper is organized as follows. Section 2 presents the theoretical
framework, Section 3 describes the material and methods used in the study,
Section 4 presents the results and discussion, and Section 5 concludes the
paper and outlines future work.

\section{Theoretical Framework}\label{sec1}

This section outlines the theoretical framework that is the basis of the study.
We begin by introducing the concept of metacognition, along with the spacing and
interweaving strategies applied in this research. Next, we present related works
organized into two subsections: research involving GenAI and studies focusing
on metacognition.

\subsection{Metacognition}

Given that GenAI often produces highly accurate and automatic responses
\citep{Puryear22}, it is essential that students use these tools within an
active learning process to enhance their learning outcomes. This active
learning process can be further understood through the lens of metacognition,
which focuses on awareness of one's mental processes. \cite{flavell79}
proposed a model of metacognitive monitoring that includes four interrelated
phenomena: Metacognitive Knowledge, Metacognitive Experience, Metacognitive
Goals, and Metacognitive Actions. These processes do not occur in isolation, but
they influence each other, altering cognitive progress over time.

Metacognitive knowledge includes beliefs about variables that affect the
outcomes of cognitive activities. It is divided into three types: beliefs about
personal abilities, perceived difficulty in the task, and previously used
strategies. Metacognitive Experience refers to the feelings that arise before,
during, and after cognitive activity, such as frustration, confusion,
satisfaction, and others. Metacognitive Goals are the key to regulating thought
as they relate to the goals the individual seeks to achieve, directly
influencing the actions taken. For example, if a student’s goal is to complete
a task quickly, they may adopt a more passive approach to learning. Lastly,
Metacognitive Actions involve the planning, monitoring, and evaluation of
strategies used to achieve the goals. In terms of planning, students can
determine how to approach a task, such as spacing their study sessions
\citep{Carvalho20}, interleaving topics \citep{Rivers21}, utilizing
retrieval practice \citep{larsen18}, and other possible strategies.

Spacing and interleaving are two metacognitive strategies that have been shown
to enhance learning outcomes and support the research questions of this study.
Spacing refers to the practice of distributing study sessions over time, which
has been shown to improve long-term retention and understanding of the material
\citep{Carvalho20}. Interleaving involves mixing different topics or problems
within a single learning session, which has been shown to also enhance long-term
learning and the application of student's knowledge in other contexts and
situations \citep{Rivers21} .

\subsection{Related Work}

\cite{Margulieux24} conducted a study on how undergraduate students in
introductory programming courses used generative AI to solve programming
problems in a naturalistic setting. The research focused on examining the
relationship between AI usage and students’ self-regulation strategies,
self-efficacy, and fear of failure in programming. Furthermore, the study
explored how these variables interacted with the characteristics of the learners,
the perceived usefulness of AI, and academic performance. The findings revealed
that students with higher self-efficacy, lower fear of failure, or higher prior
grades tended to use AI less frequently or later in the problem-solving process
and perceived it as less useful compared to their peers. However, no significant
relationship was found between students’ self-regulation strategies and their
use of AI.

% \cite{Boudouaia24} investigated the impact of ChatGPT-facilitated programming on
% college students’ programming behaviors, performance, and perceptions. The
% problem addressed was whether the use of ChatGPT in programming education could
% influence how students interacted with programming tasks, their learning
% outcomes, and their attitudes towards technology. The study employed a
% quasi-experimental design involving 82 college students divided into two groups:
% one using ChatGPT-assisted Programming (CAP) and the other using traditional
% self-directed Programming (SDP). The method combined mixed data collection
% techniques, including behavioral logs, programming performance evaluations, and
% interviews. The results showed that students using the CAP mode exhibited more
% frequent behaviors such as debugging and reading feedback. Although CAP students
% had a higher average score in programming performance, there was no
% statistically significant difference compared to the SDP group. However,
% students’ perceptions of ChatGPT improved significantly after the intervention,
% with increased perceived usefulness, ease of use, and intention to continue
% using the tool in the future.

The study of \cite{Boudouaia24} examined the effects of ChatGPT-assisted
programming on university students' behaviors, performance, and perceptions.
A quasi-experimental research was conducted with 82 students divided into two
groups: one with ChatGPT-assisted programming (CAP) and the other with
self-directed programming (SDP). The analysis included behavioral logs,
performance evaluations, and interviews. Students in the CAP group engaged more
actively in debugging and feedback review activities. Although they achieved
slightly higher scores, there was no statistically significant difference in
performance compared to the SDP group. Nevertheless, perceptions of ChatGPT
improved significantly, highlighting greater perceived usefulness, ease of use,
and intention to use the tool in the future.

\cite{Bastani24} investigated the impact of GenAI, specifically GPT-4, on human
learning, with a focus on mathematics education in a high school. The problem
addressed was how the use of generative AI could affect the acquisition of new
skills, which is crucial for long-term productivity. The method involved a
controlled experiment with approximately one thousand students, who were exposed
to two GPT-4-based tutors: a simple tutor (GPT Base) and another with safeguards
designed to promote learning (GPT Tutor). The results showed that while access
to GPT-4 improved performance on practice exercises (48\% with GPT Base and
127\% with GPT Tutor), the removal of access to GPT Base led to a 17\% decrease
in student performance on exams. This suggests that unrestricted use of
\textit{GPT Base} could hinder learning. However, the GPT Tutor was able to
mitigate this negative effect.

\subsection{Metacognition Related Works}

Learning based on metacognitive strategies has been shown to be effective in
enhancing students’ performance and skills. \cite{Zheng19} investigated the
effects of metacognitive scaffolds on group behavior, performance, and cognitive
load in computer-supported collaborative environments. The results showed
positive impacts on metacognitive behavior and group performance without
increasing the cognitive load. \cite{LiWei23} proposed a metacognition-based
collaborative approach to enhance performance in collaborative programming. The
findings indicated that this approach influenced computational thinking,
critical thinking, and metacognitive awareness. Similarly, \cite{Wang23}
explored how metacognitive instruction affects computational thinking, critical
thinking, and metacognitive skills in collaborative programming, observing

\cite{Khusnul24} investigates
the efficacy of a meta-learning approach in improving metacognitive and creative
skills. This study confirms the effectiveness of metacognition strategies and
elucidates the relationship between meta-learning and metacognition. This
research suggests that meta-learning can improve metacognitive abilities,
providing valuable insights into educational technology and course design in
higher education settings.

\cite{Sidra24} addressed the growing importance of metacognitive skills in an
AI-enabled workforce. The problem discussed was how AI systems, especially in
collaborative tools based on Large Language Models (LLMs), impacted human
decision-making and the challenges they posed due to the lack of human-like
metacognition in AI. The method used identified four main characteristics that
differentiated human-AI interaction from human-human collaboration and proposed,
based on the dual-process theory of human thought, that metacognitive thinking
was crucial for the effective use of AI. The results indicated that human
workers needed to develop and apply metacognitive skills, such as planning,
monitoring, and evaluating their thought processes, to mitigate cognitive biases
and improve decision-making when working with AI. The study concluded by
recommending improvements in AI design and training to improve workers'
metacognitive abilities, ensuring more effective collaboration with AI systems.

\cite{Sidra24} explored the importance of metacognitive skills in an AI-enabled
workforce, focusing on how AI, particularly tools using Large Language Models,
affects human decision-making. They highlighted challenges arising from AI's
lack of human-like metacognition and identified key differences between human
AI and human-human collaboration. Using dual-process theory, they emphasized the
need for workers to develop metacognitive skills like planning, monitoring, and
evaluating to counter cognitive biases and enhance decision-making. The study
recommended improving AI design and training to boost workers' metacognitive
abilities for more effective collaboration.

\section{\uppercase{Material and Method}}

This section outlines the tools and procedures used in this study. CharlieBot
served as the GenAI tool for educational purposes, and the method was structured
into three distinct phases, each designed to systematically assess its
interaction with students.

\subsection{CharlieBot}

CharlieBot is an educational chatbot powered by ChatGPT 3.5 and enhanced with
Retrieval-Augmented Generation (RAG) \citep{Sun24}. RAG is an AI technique that
integrates information retrieval with generative models. It first retrieves
relevant documents or data from a knowledge base or external source using a
retrieval model. Then, a generative model uses this retrieved information to
generate more accurate, context-aware responses. This dual-step process enhances
the quality and reliability of the chatbot's answers. A prior study on
CharlieBot's performance revealed that most students found its responses
appropriate for the Java programming course \citep{Hoyos24}.

\begin{table*}[htbp]
  \caption{Categories - adapted from \cite{Ghimire24}}
  \begin{center}
    \renewcommand{\arraystretch}{1.2} % Increase the spacing between rows
    \begin{tabular}{p{4.1cm} p{4cm} p{3.7cm}} % Remove vertical bars
      \hline
      \textbf{Category} & \textbf{Description} & \textbf{Example of prompt} \\
      \hline
      Debugging Help (DH) & Prompts that seek help to identify, fix errors, or understand the provided code snippet. & \textit{Would this code be ok? \{code\}} \\
      Conceptual Question (CQ) & Prompts that are more about understanding concepts than specific code. & \textit{What does it mean for a method to be static?} \\
      Student Correction (SC) & Prompts where the student corrects the chatbot. & \textit{The correct answer is B} \\
      Code Snippet  (CS) & Prompts that ask for a specific part of the code, like a function or a segment. & \textit{A class inherits from another write this code} \\
      Complete Solution (CSO) & Prompts that request an entire solution or a complete code snippet. & \textit{Give me the code for a selection sort} \\
      Multiple Question (MQ) & Prompts where the user wants to solve a multiple choice exercise (Quiz). & \textit{A heap is a data structure appropriate for: \{options\}} \\
      Language Change (LC) & Prompts that request a change of idiom. & \textit{In Spanish} \\
      Uncategorized (U) & Prompts that do not fit into any of the above categories. & \textit{Thanks} \\
      \hline
    \end{tabular}
    \label{tab:categories}
  \end{center}
\end{table*}

\subsection{Method}

The study comprised three phases: data collection, categorization, and analysis.
During the data collection phase, students enrolled in a second-semester Java
course at the University Carlos III of Madrid (UC3M) were introduced to
CharlieBot and allowed to use it without following any prescribed educational
methodology. All data were collected anonymously to ensure that interactions
could not be traced back to individual students or linked to their academic
performance.

During the categorization phase, the students' prompts were initially classified
into eight distinct categories using the Claude.ai \citep{claude} tool.
Subsequently, the authors of this study manually reviewed these classifications
to ensure accuracy and alignment with the research objectives.
Initially, \cite{Ghimire24} proposed four categories: Debugging Help (DH),
Conceptual Question (CQ), Code Snippet (CS), and Complete Solution (CSO).
However, the data collected indicated the need for additional categories,
leading to the inclusion of four more: Multiple Questions (MQ), Student
Corrections (SC), Language Change (LC), and Uncategorized (U). Unlike study
\cite{Ghimire24}, which analyzed students' behavior using artificial
intelligence for specific programming tasks, our study allowed participants to
explore the chatbot as they preferred. This approach enabled the emergence of
additional categories of analysis, broadening the understanding of the tool's
uses and interactions. Table \ref{tab:categories} presents these categories,
their descriptions, and an example. Besides that, the topics covered in the
students' prompts were categorized based on the course issues, which include
(1) Java, (2) Object Orientation, (3) Testing, (4) Recursion, (5) Data
Structures, and (6) Sorting and Searching Algorithms.

The analysis phase involved using Python/Pandas scripts to extract information
from previously classified data.


% Ensure the bibliography style and data are correctly included
\bibliography{sn-bibliography}

\end{document}
