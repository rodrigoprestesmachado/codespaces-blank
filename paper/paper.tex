\documentclass[conference]{IEEEtran}
\IEEEoverridecommandlockouts
% The preceding line is only needed to identify funding in the first footnote. If that is unneeded, please comment it out.
\usepackage{cite}
\usepackage{amsmath,amssymb,amsfonts}
\usepackage{algorithmic}
\usepackage{graphicx}
\usepackage{textcomp}
\usepackage{xcolor}
\usepackage{array}
\usepackage{multirow}
\def\BibTeX{{\rm B\kern-.05em{\sc i\kern-.025em b}\kern-.08em
    T\kern-.1667em\lower.7ex\hbox{E}\kern-.125emX}}
\begin{document}

\title{An In-Depth Analysis of Student Prompts With CharlieBot \\
{\footnotesize \textsuperscript{*}Note: Sub-titles are not captured in Xplore and
should not be used}
\thanks{Identify applicable funding agency here. If none, delete this.}
}

\author{\IEEEauthorblockN{1\textsuperscript{st} Rodrigo Prestes Machado}
\IEEEauthorblockA{\textit{dept. Informática} \\
\textit{Instituto Federal do Rio Grande do Sul}\\
Porto Alegre, Brazil \\
0000-0003-0428-6387}
\and
\IEEEauthorblockN{2\textsuperscript{nd} Given Name Surname}
\IEEEauthorblockA{\textit{dept. name of organization (of Aff.)} \\
\textit{name of organization (of Aff.)}\\
City, Country \\
email address or ORCID}
\and
\IEEEauthorblockN{3\textsuperscript{rd} Given Name Surname}
\IEEEauthorblockA{\textit{dept. name of organization (of Aff.)} \\
\textit{name of organization (of Aff.)}\\
City, Country \\
email address or ORCID}
\and
\IEEEauthorblockN{4\textsuperscript{th} Given Name Surname}
\IEEEauthorblockA{\textit{dept. name of organization (of Aff.)} \\
\textit{name of organization (of Aff.)}\\
City, Country \\
email address or ORCID}
\and
\IEEEauthorblockN{5\textsuperscript{th} Given Name Surname}
\IEEEauthorblockA{\textit{dept. name of organization (of Aff.)} \\
\textit{name of organization (of Aff.)}\\
City, Country \\
email address or ORCID}
\and
\IEEEauthorblockN{6\textsuperscript{th} Given Name Surname}
\IEEEauthorblockA{\textit{dept. name of organization (of Aff.)} \\
\textit{name of organization (of Aff.)}\\
City, Country \\
email address or ORCID}
}

\maketitle

\begin{abstract}
This document is a model and instructions for \LaTeX.
This and the IEEEtran.cls file define the components of your paper [title, text, heads, etc.]. *CRITICAL: Do Not Use Symbols, Special Characters, Footnotes, 
or Math in Paper Title or Abstract.
\end{abstract}

\begin{IEEEkeywords}
component, formatting, style, styling, insert
\end{IEEEkeywords}

\section{Introduction}

%What is the problem to be solved?

According \cite{LO2024105100} further research about the use of AI tools in
education is needed to understand how these tools can be used to improve
students' learning experience. In particular, the study suggested that future
research on students' study habits.


% Are there any existing solutions?

% Which is the best?

% What is its main limitation?

% What do you hope to achieve?

\section{Background}

The paper \cite{10343037} examined students from different demographic groups
using AI tools like ChatGPT, comparing perceptions and impacts
among these groups. The results showed that AI tools had a positive impact on
academic performance, especially for students with less prior experience.

\begin{table*}[htbp]
\caption{Categories of Analysis}
    \begin{center}
    \begin{tabular}{|p{3cm}|p{5cm}|p{3cm}|}
    \hline
    \textbf{Category} & \textbf{Description} & \textbf{Critical Thinking} \\
    \hline
    Debugging Help & Prompts that seek help to identify, fix errors, or understand the provided code snippet. & \multirow{6}{3cm}{Tends to promote.} \\
    Conceptual Questions & Prompts that are more about understanding concepts or algorithms rather than specific code. & \\
    Student Correction & Prompts where the student corrects the bot. & \\
    \hline
    Code Snippet & Prompts that ask for a specific part of the code, like a function or a segment. & \multirow{6}{3cm}{Tends not to promote.} \\
    Complete Solution & Prompts that request an entire solution or a complete code snippet. & \\
    Multiple Question & Prompts where the user wants to solve a multiple-question exercise. & \\
    \hline
    Language Change & Prompts that request a change of idiom to convey a message more effectively. & \multirow{3}{3cm}{Neutral.} \\
    Uncategorized & Prompts that do not fit into any of the above categories. & \\
    \hline
    \end{tabular}
    \label{tab:categories}
    \end{center}
\end{table*}

The table \ref{tab:categories} shows the categories of analysis used in this
study. The categories were based on the work of
\cite{10.1007/978-3-031-64299-9_20}, with the addition of four new categories.
Also, we group the categories to understand the impact on critical thinking.


\section{Method}

CharlieBot is a ChatGPT-based bot that provides assistance to students in
Java programming courses in UC3M. CharlieBot logs all conversations with
students, which were used for this study. The goal was to categorize student
messages to understand how students interact with the bot and identify patterns
in their study habits.
\\
The categories outlined in \cite{10.1007/978-3-031-64299-9_20} served as the
foundation for classifying student messages. However, it was identified that
some categories were missing, necessitating the addition of four new categories.
Table \ref{tab:categories} presents these categories along with their
descriptions.
\\
Subsequently, these eight categories were input into Claude.ai, accompanied by
detailed instructions on the desired structure of the input and the expected
format of the output. For each conversation with the CharlieBot log, a .csv file
was downloaded, and Claude was tasked with classifying the user messages into
one of the eight categories.

The classifications made by Claude were then manually reviewed. In cases where
there was disagreement with the classification, manual adjustments were made,
and the accuracy of Claude's classification was recorded. Each conversation
with CharlieBot was saved into a .csv file containing the original conversation,
along with two additional columns: "Classification" to denote the category and
"AI" to indicate Claude's accuracy.

As new categorizations were completed, an example was submitted to Claude's
knowledge base to enhance accuracy. Initially, Claude began with an example
representing 1\% of the total memory of its knowledge base. As classifications
were progressively completed, more examples were provided to Claude to increase
the percentage of its knowledge base occupied by relevant examples.

Upon completion of all categorizations, all resulting .csv files were consolidated
for final counts and analysis using the Pandas library. This methodology
facilitated an efficient categorization of student messages, thereby enabling
a comprehensive analysis of how tools such as ChatGPT are utilized in foundational
programming courses.

\section{Results and Discussion}

The graph \ref{fig:graph1} shows the distribution of the messages in the
categories:
\begin{figure}[h!]
    \centering
    \includegraphics[scale=0.7]{figures/figure1.png}
    \caption{Classification of messages.}
    \label{fig:graph1}
\end{figure}

Approximately 64\% of the messages pertain to categories associated with
critical thinking, corroborating with the findings of
\cite{10.1007/978-3-031-64299-9_20}. In contrast, around 28\% of the messages
indicate a preference among students for ready-made answers.

\section{Conclusion}

\bibliographystyle{IEEEtran}
\bibliography{references}

\end{document}
